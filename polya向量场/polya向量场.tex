\documentclass[hyperref,UTF8]{ctexart}
\usepackage[dvipdfmx]{graphicx}
\usepackage{gbt7714}
\usepackage{float}
\usepackage{ragged2e}
\usepackage{amsthm}
\usepackage{amssymb}
\usepackage{amsmath}
\usepackage{wrapfig}
\usepackage{tikz}
\usetikzlibrary{arrows.meta}
\usepackage{booktabs}
%\usepackage[a4paper,left=3.18cm,right=3.18cm,top=2.54cm,bottom=2.54cm]{geometry}
\usepackage{tabularx}
\usepackage{array}
\usepackage{caption}
\usepackage{hyperref}
\newcommand{\upcite}[1]{\textsuperscript{\textsuperscript{\cite{#1}}}}
\setCJKfamilyfont{song}{SimSun}
\title{polya向量场}
\author{limbo137}
\begin{document}
对于复函数 $w=u+iv$可以看作在 $(x,y)$处具有下面的向量场,
\[\begin{pmatrix}
    u\\v
\end{pmatrix}\]  
但这样的向量场的性质往往不尽如人意,但是我们对于复数,有下面两种导数的定义
\[\begin{aligned}
    \frac{\partial}{\partial z} &=\frac{\partial x}{\partial z} \frac{\partial}{\partial x}+\frac{\partial y}{\partial z} \frac{\partial}{\partial y} \\
    &=\frac{\partial}{\partial x}-i \frac{\partial}{\partial y}:=\partial
    \end{aligned}
\]
以及
\[\begin{aligned}
    \frac{\partial}{\partial \bar{z}} &=\frac{\partial x}{\partial \bar{z}} \frac{\partial}{\partial x}+\frac{\partial y}{\partial \bar{z}} \frac{\partial}{\partial y} \\
    &=\frac{\partial}{\partial x}+i \frac{\partial}{\partial y}:=\bar{\partial}
    \end{aligned}\]
注意到解析函数天然地满足
\[\partial \bar{w}=\bar{\partial} w=0\]
对于任意可导函数,
\[\begin{aligned}
    \partial \bar{w} &=\left(\frac{\partial}{\partial x}+i \frac{\partial}{\partial y}\right)(u-i v) \\
    &=\frac{\partial u}{\partial x}+\frac{\partial v}{\partial y}+i\left(\frac{\partial v}{\partial x}-\frac{\partial u}{\partial y}\right)
    \end{aligned}
\]
\end{document}