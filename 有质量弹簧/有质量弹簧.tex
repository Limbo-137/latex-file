\documentclass[hyperref,UTF8]{ctexart}
\usepackage[dvipdfmx]{graphicx}
\usepackage{gbt7714}
\usepackage{float}
\usepackage{ragged2e}
\usepackage{amsthm}
\usepackage{amssymb}
\usepackage{amsmath}
\usepackage{wrapfig}
\usepackage{tikz}
\usetikzlibrary{arrows.meta}
\usepackage{booktabs}
%\usepackage[a4paper,left=3.18cm,right=3.18cm,top=2.54cm,bottom=2.54cm]{geometry}
\usepackage{tabularx}
\usepackage{array}
\usepackage{caption}
\usepackage{hyperref}
\newcommand{\upcite}[1]{\textsuperscript{\textsuperscript{\cite{#1}}}}
\setCJKfamilyfont{song}{SimSun}
\title{有质量弹簧}
\author{limbo137}
\begin{document}
\maketitle
\section{离散弹簧的连续化}
\begin{enumerate}
    \item \(u_i\):第\(i\)个质点的坐标
    \item \(\varepsilon\):每个质点之间原长的间隔
\end{enumerate}
\subsection{自由情况}
设$\kappa = kl$,每一个\(\varepsilon\)的弹性系数$=nk=kl/\varepsilon=\kappa/\varepsilon$,其中\(n=L/\varepsilon\)

利用牛顿方程,我们有
\[\frac{\kappa}{\varepsilon}(u_{i+1}-u_{i}-\varepsilon)-\frac{\kappa}{\varepsilon}(u_i-u_{i-1}-\varepsilon)=\lambda\varepsilon\ddot{u_i}\]
连续化,有
\[\kappa(\frac{\partial u}{\partial x}|_{x+\varepsilon}-\frac{\partial u}{\partial x}|_{x})=\lambda\varepsilon\ddot{u}\]
则有
\[\frac{\partial^2 u}{\partial x^2}=\frac{\lambda}{\kappa}\frac{\partial^2 u}{\partial t^2}\]
设\(u=A_i e^{i(\omega t-kx)}\)
有\[k^2=\frac{\lambda}{\kappa }\omega^2\]
\end{document}