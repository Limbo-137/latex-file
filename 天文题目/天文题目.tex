\documentclass[hyperref,UTF8]{ctexart}
\usepackage[dvipdfmx]{graphicx}
\usepackage{gbt7714}
\usepackage{float}
\usepackage{ragged2e}
\usepackage{amsthm}
\usepackage{amssymb}
\usepackage{amsmath}
\usepackage{wrapfig}
\usepackage{tikz}
\usetikzlibrary{arrows.meta}
\usepackage{booktabs}
%\usepackage[a4paper,left=3.18cm,right=3.18cm,top=2.54cm,bottom=2.54cm]{geometry}
\usepackage{tabularx}
\usepackage{array}
\usepackage{caption}
\usepackage{enumerate}
\usepackage{hyperref}
\newcommand{\upcite}[1]{\textsuperscript{\textsuperscript{\cite{#1}}}}
\setCJKfamilyfont{song}{SimSun}
\title{天文题目}
\date{}
\begin{document}

\subsection*{第1题} 
下列属于天文学研究范畴的是 

\begin{enumerate}[\label= A.]
    \item 恒星的演变
    \item 牛郎和织女的爱恨情仇 
    \item 朝晖夕阴,气象万千 
    \item 你这星星多少钱一斤?
\end{enumerate}
\subsection*{第2题} 
在天空中,可以被人眼看到的星星不包括以下哪个 
\begin{enumerate}[\label= A.]
    \item 流星 
    \item 哈雷彗星
    \item 氮星(Krypton)
    \item 太阳
\end{enumerate} 

\subsection*{第3题}
在地球附近的天体中对地球潮汐影响最大的是 
\begin{enumerate}[\label= A.]
    \item 太阳
    \item 月亮
    \item 木星
    \item 银河中心黑洞
\end{enumerate} 

\newpage

\subsection*{第4题} 
恒星的演化不包括下列哪个阶段

\begin{enumerate}[\label= A.]
    \item 红巨星
    \item 黑洞
    \item 白矮星
    \item 矮行星
\end{enumerate}

\subsection*{第5题} 下列哪个定理或理论常被用来计算行星的运动轨迹 \begin{enumerate}[\label= A.]
    \item 黑暗森林理论 \item 剩余价值理论 \item 广义相对论及牛顿运动定律\item 雷氏运动力学
\end{enumerate} 

\subsection*{第6题} 
下列那本著作与天文最相关
\begin{enumerate}[\label= A.]
    \item 《孙子兵法》\item 《梦溪笔谈》\item 《甘石星经》\item 《母猪的产后护理》
\end{enumerate} 
\newpage
\subsection*{第7题}
下列发生在月亮上的故事,符合事实的是
\begin{enumerate}[\label= A.]
    \item 第二次月面战争\item 希特勒逃往月球背面并建立基地\item 嫦娥奔月\item 阿波罗11号登上月球
\end{enumerate} 

\subsection*{第8题} 
地球到太阳的距离大约是
\begin{enumerate}[\label= A.]
    \item 8光秒\item 8光分\item 8光时\item 8光年
\end{enumerate}

\subsection*{第9题} 
天文单位指
\begin{enumerate}[\label= A.]
    \item 与日常单位进率较大的几种单位\item 一种时间单位\item 一种长度单位\item 一种质量单位
\end{enumerate} 
\newpage 
\subsection*{第10题} 
暗物质和暗能量的关系是\begin{enumerate}[\label= A.]
    \item 暗物质是暗能量的载体\item 它们是一种东西\item 它们是cp \item 它们没什么关系
\end{enumerate} 



\subsection*{第11题} 下列哪个不属于天文常用的单位
\begin{enumerate}[\label= A.]
    \item 天文单位\item 光年\item 秒差距\item 干米
\end{enumerate} 

\subsection*{第12题} 天空中有多少个星座?
\begin{enumerate}[\label= A.]
    \item 36个\item 72个\item 12个\item 88个
\end{enumerate} 
\newpage
\subsection*{第13题} 赫罗图是指\begin{enumerate}[\label= A.]
    \item 赫敏和罗恩的合影\item 赫拉克勒斯和肯泰罗的合影\item 根据恒星的光度随温度的关系所绘制的图\item 描述粒子相互作用的一种图
\end{enumerate} 

\subsection*{第14题} 
下列哪种是天文上测量所测天体的距离的方法
\begin{enumerate}[\label= A.]
    \item 凝胶色谱柱法\item len()方法\item 消视差法\item 三角视差法
\end{enumerate} 

\subsection*{第15题} 太阳光谱中,哪个颜色的波段的光最强?
\begin{enumerate}[\label= A.]
    \item 白光
    \item 紫光
    \item 绿光
    \item 红光
\end{enumerate}
\newpage
\subsection*{第16题} 月球绕地球旋转一圈的周期大概是
\begin{enumerate}[\label= A.]
    \item 1天
    \item 5天
    \item 10天
    \item 30天
\end{enumerate}

\subsection*{第17题} 月球的一天大概相当于地球的
\begin{enumerate}[\label= A.]
    \item 1天
    \item 5天
    \item 10天
    \item 30天
\end{enumerate}

\subsection*{第18题} 地球与月球的距离为
\begin{enumerate}[\label= A.]
    \item 1光年
    \item 1光秒
    \item 384399公里
    \item 14710万公里
\end{enumerate}
\newpage
\subsection*{第19题} 最近一次发射的探月火箭为
\begin{enumerate}[\label= A.]
    \item 嫦娥一号
    \item 嫦娥二号
    \item 嫦娥三号
    \item 嫦娥五号
\end{enumerate}

\subsection*{第20题} 为什么中秋的月亮最大最圆
\begin{enumerate}[\label= A.]
    \item 因为嫦娥此时最开心
    \item 因为它离地球最近
    \item 因为它吸收到的阳光最多
    \item 因为它此时能量最大
\end{enumerate}
\end{document}