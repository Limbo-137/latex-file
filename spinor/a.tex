\documentclass[utf8]{article}
\usepackage{amsmath}
\begin{document}
我们最后指出,任何纯boost(两个与$\mathbf{t}$垂直的超平面保持不变,例如上面的$X =0, Y =0$)对应于一个正的或负的厄米自旋矩阵,反之亦然。对z-boost(\ref{eq:1237})就是这种形式,为了获得其他方向的boost,我们只需要将这个方向旋转到$z$方向,应用z-boost,然后旋转回来。这对应于自旋矩阵$\mathbf{A}^{-1}\mathbf{B}\mathbf{A}$,其中$\mathbf{A}$是所需的旋转,$\mathbf{B}$是z-boost;根据初等矩阵理论,$\mathbf{A}^\dagger \mathbf{B}\mathbf{A}$仍然是正负定的厄米矩阵。相反地,任何正负定厄米矩阵$\tilde{\mathbf{B}}$都可以由一个酉矩阵$\mathbf{A}$对角化:$\mathbf{A}\tilde{\mathbf{B}}\mathbf{A}^{-1}= \text{diag}(\alpha,\delta)$,其形式必须是$\pm \text{diag}(w^{\frac12},w^{-\frac12})=±\mathbf{B}$,因为厄米性、定性和单位行列式得以保留。因此$\tilde{\mathbf{B}}$的形式为$\pm \mathbf{A}^{-1}\mathbf{B}\mathbf{A}$,我们的结果成立。
\end{document}