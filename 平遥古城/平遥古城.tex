\documentclass[UTF8]{ctexart}
\usepackage{graphicx}
\usepackage{gbt7714}
\usepackage{float}
\usepackage{ragged2e}
\usepackage{amsthm}
\usepackage{amssymb}
\usepackage{amsmath}
\usepackage{wrapfig}
\usepackage{booktabs}
\usepackage[a4paper,left=3.18cm,right=3.18cm,top=2.54cm,bottom=2.54cm]{geometry}
\usepackage{tabularx}
\usepackage{array}
\usepackage{caption}
\newcommand{\upcite}[1]{\textsuperscript{\textsuperscript{\cite{#1}}}}
\setCJKfamilyfont{song}{SimSun}
\newcommand{\xiaosihao}{\fontsize{12pt}{\baselineskip}\selectfont}


\title{基于游客体验视角下的平遥古城保护和发展提案}
\author{古陶虎探队\footnote{导师:杨锋梅,队长:高弘,队员:周和红,韩彤,赵翔,程泉}}
\date{}
\begin{document}
\maketitle
\tableofcontents
\begin{abstract}
    平遥古城是目前我国唯一以整座古城申报世界文化遗产获得成功的古县城,既有历史的厚重感,又兼备现代文明的活力。传统的文化遗产保护形式已经不能满足平遥古城现状的发展需求,且在疫情常态化的时代背景下,古城的保护和利用需要有所转变。因此,本文基于“游客体验”的视角,将古城的保护和利用作为研究对象,采用李克特五分量表法对实地调查的问卷进行分析,获取了游客对古城的游玩体验,作出一定的评估,以此探寻古城保护与发展的新对策。并针对所发现的问题,提出相应的解决方案,为平遥古城在新时期的保护和发展提供研究基础。\\
    \textbf{关键词:}平遥古城 \quad 游客体验 \quad 文化遗产 \quad 旅游业 \quad 保护与利用
\end{abstract}
\section{引言}
    \subsection{研究背景}
    古圣前贤留遗珠,今才后杰筑新城。1986 年, 国务院公布平遥古城为国家历史文化名城,1997 年平遥古城被列入世界文化遗产名录,2009 年明清街(南大街)入选首批“中国历史文化名街”。近年来,随着游客对旅游质量要求的不断提高以及疫情常态化对旅游业的冲击,古城需要更具时代性的保护发展对策。游客是景区活化的动力源,是景区在保护发展过程中必须关注的一部分,因此让游客在游玩的过程中感受到宾至如归,在古城保护发展中变得极其重要。基于游客体验视角,对平遥古城保护与利用提出可持续性的策略,是本文研究的重点问题。
    \subsection{研究范围}
    2021年3月,团队成员在平遥古城景区内进行了实地考察、调查问卷的分发与回收,以及对居民、商铺和游客的访谈工作,考察包括古城建筑、卫生、服务、风土人情、基础设施、绿化状况、居民生活、非物质文化遗产以及文化创意产品等诸多影响游客体验的方面进行了调查和评估。
    \subsection{研究方法}
    团队首先采用的是实地调查和问卷调查的方式,通过亲身体验和数据分析(主要是李克特五分量表法),以及对平遥古城相关群体的访谈结果进行分析,从而得出基于游客体验角度的整体发展分析和对策。
\section{平遥古城文化遗产开发赋存}
平遥古城位于山西省的中部、晋中地区的南部,是一座具有2700多年历史的文化名城。 在历史上,古城春秋时属晋国,战国时属赵国。秦在此置平陶县,汉置中都县,为宗亲代王的都城。北魏时改名为平遥县。平遥曾是清代晚期中国的金融中心,并有中国目前保存最完整的古代县城格局。清代晚期,总部设在平遥的票号就有二十多家,占全国的一半以上,因此也被称为“古代中国华尔街”\upcite{秦晋2012平遥以中国}。其中规模最大的是创建于清道光年间、以“汇通天下”而闻名于世的中国第一座票号“日升昌”。平遥古城经济以农业为主,主产粮食、棉花,特产牛肉、推光漆器等。其中牛肉名声颇大,有“平遥牛肉太谷饼”的民歌歌词。

龟城巍峨,捍卫一方。两千多年风吹雨打,千万烽火明华不灭。在漫长的发展中,平遥古城保留的文化遗存数量之多、密度之高、跨度时间之长,使它在被誉为“中国古建筑宝库”的山西省境内,亦有“文物大县”之称。人们认为平遥古城的优点也集中于“古、全、真”三个方面。平遥古城众多的文化遗存,不仅可以展现中国古代城市在不同历史时期的建筑形式、施工方法和用材标准,也集中体现了公元 14 至 19 世纪前后汉民族的历史文化特色,对研究这一时期的社会形态、经济结构、军事防御、传统思想等有着重要的参考价值。
    \subsection{平遥古城文化遗产的种类}
平遥县历经千年风雨,人文底蕴深厚,非物质文化遗产恒河沙数(见图)。各种类型的文化遗产都可以打造成古城特有的产品。
\begin{figure}[H]
    \centering
    \caption{平遥县非物质文化遗产目录}
    \includegraphics[width=10cm]{非物质文化遗产.png}
    \label{fig:my_label}
\end{figure}
\begin{figure}
    \centering
    \caption{平遥物质文化遗产}
    \includegraphics[width=8cm]{物质文化遗产.png}
    \label{fig:my_label}
\end{figure}
平遥县历史悠久,先贤给平遥留下了宝贵的物质财富和精神财富,既有古建筑古物品方面,又有很大的非物质文化印记。平遥县非物质文化遗产浩如烟海,种类繁多:不仅上可列入国家级省级,下至市级县级;又囊括语言,民间美术和民俗等各遗产种类。随着社会经济的发展,年轻群体越来越对非物质文化燃起了极大的兴趣和热情。基于此类市场群体的扩大和平遥文化遗产吸引力的增强。
    \subsection{平遥古城文化遗产保护与利用概况}
1997年,平遥古城被列入世界文化遗产目录,与同为第二批国家历史文化名城的四川阆中、云南丽江、安徽歙县并称为“保存最为完好的四大古城”。

近年,平遥古城以古城墙,古街道,古建筑的保护为前提,秉持着“保护为主,抢救第一,合理利用,加强管理”的发展理念,成功树立 了 以古城为核心的旅游景区 品 牌。据了解,全县形成了 以“一城两寺”24处景点、 6条特色街区(东、西、南、北大街,城隍庙街,衙门街) 、 3个演艺(又见平遥、晋商乡 音、迎宾仪式)为重点的旅游产品 体系,以文化资源为依托的文化旅游产业快速发展,平遥古城也成为全省乃至全国极具知名度的旅游胜地。目前,平遥古城内现有各级重点文物保护单位28处,其中国家级文保单位7处,市级文保单位2处,县级文保单位19处,文物古迹众多。除传统旅游资源外,现代节庆展览等文化旅游已成为平遥最具活力和最具发展潜力的朝阳产业。
\begin{table}[H]
    \centering
    \caption{平遥古城部分节庆展览活动}
    \begin{tabular}{p{4cm}p{4cm}p{4cm}}
        \toprule
        节庆 & 举办年份 & 主题\\
        \midrule
        平遥国际雕塑节 & 2018第一届 &“中西方艺术文化交流”\\
        &2019第二届&“之间”\\
        &2020第三届&“生态美学”\\
        \cmidrule{1-3}
        平遥中国年&2018&“欢歌迎春”、“翰墨同春”、“彩灯靓春”、“社火闹春”、“民俗乐春”、“网络贺春”、“戏曲唱春”七个板块\\
        \cmidrule{1-3}
        平遥摄影大展&2016&“天地心·家国情”\\
        &2017&“回望初心·梦幻未来”\\
        &2019&“守正·创新”\\
        \bottomrule
    \end{tabular}
\end{table}
然而与此同时,在新冠疫情影响之下,病毒传播速度快,辐射地域广,对涉及人员流动的旅游业来说,影响也愈加广泛和深刻。在2020年初新冠疫情到来时,平遥古城自觉关闭所有商铺门店,所有的旅游从业人员暂离岗位,景区经济收入下滑严重。世界遗产平遥古城———如何在疫情的冲击下,发挥维持地方社会经济发展中的重要作用成为当务之急。如今正处于疫情防控后期阶段,景区采用了限制游客最大承载量、要求游客佩戴口罩、提供健康码以及保持1.5米间隔距离等通用安全防护措施,对古城旅游采用免门票的短期措施,以此来刺激旅游和经济的恢复。然而,在一系列促进旅游恢复的措施之下,平遥古城应对疫情冲突的特有措施仍存在诸多问题。
\begin{table}[H]
    \centering
    \caption{平遥古城游客接待量数据大致汇总(单位:万人次)}
    \begin{tabular}{p{2cm}p{2cm}p{2cm}p{2cm}p{2cm}p{2cm}}
        \toprule
        年份 & 国庆&清明&端午&春节&元旦\\ 
        \midrule
        2019& & & 11.52 & 45.39 & 1.6\\ 
        2018& 62.8 & 11.65 & 11.25 & 43.01 & 1.59 \\ 
        2017& 61.1 & 11.30 & & & \\ 
        \bottomrule
    \end{tabular}
\end{table}
\section{古城文化遗产保护与利用现状的分析}
    \subsection{古城环境保护现状}
        \subsubsection{古城施工所造成的环境污染}
\textbf{(1).噪音污染问题:}政府为了保护文物,采取了一系列的修复方案,但在建设过程中,会引起众多的实际问题,其中最为突出的还是噪声污染。古城内的道路修缮工程紧靠居民区,在工程项目施工的过程中,设施机器发出的声音会直接影响周围居民,同时对施工人员本身也具有极大的危害性。

\textbf{(2).粉尘、固体垃圾污染问题:}大气污染问题也是建筑工程施工过程中的突出问题之一。在古城道路修缮和双林寺修缮施工过程中,施工材料的来回运输以及搅拌过程、砂石料的搅拌等等都会造成一定的灰尘,这些灰尘就会随着气流向周围扩散,会形成一定的粉尘大气污染,粉尘污染也是导致施工人员职业病的重要根源之一,同时也会给周围群众的生活和出行带来不便。在古城道路中有一些建筑材料的堆积,对环境有一定的危害,且古城巷窄,建筑材料堆积对古城内交通有一定的影响。

\textbf{(3).水污染问题:}古城建筑施工时,施工设备的清洗过程会带来的水污染及水资源浪费,且很多原材料需要经过清洗或者混合,废水如果没有经过专门的处理,很容易将对环境有害的物质排出,从而引起环境污染。
        \subsubsection{空气}
平遥古城内的供暖不是集中供暖,而是独立供暖,这就导致了平遥古城内的商铺和居民采用不同的供暖方式。其中,商铺采用的是暖气取暖。另外,通过与本地的居民交流,我们了解到到本地居民采用烧煤取暖,也有部分燃烧秸秆的情况。然而本地政府认识到煤炭燃烧对环境的污染较为严重,所以部分居民转而用秸秆来取暖,本地部分居民认为秸秆易燃,燃烧秸秆产生的烟少。这说明了当地居民对环境保护意识淡薄,以及平遥古城管理者对这方面监管不力。
        \subsubsection{绿化}
平遥古城的古城墙外部是砖瓦堆积,内部是土方,由于年代原因以及恶劣天气的种种因素,平遥古城的内部土方从城墙顶部往向下流,呈现边坡的状况,目前采用的方法是填土护坡法,但是这种方法需要大量的资金来购买适宜的土方从而来修补城墙,而且这种方法在后期的维护中需要重复地修补,这不仅影响景观观赏度,还有可能在天气恶劣时造成人员伤亡,现在大多数古城采用的都是古城墙绿化的方法,这种方法相较于填土护坡法的前期投入成本高,但是在后期的养护中可以一劳永逸,并且这种方式的景观观赏度高,可以增加古城的景观价值,古城墙绿化还可以减少水土流失,防止顶部的土再流到下方。
        \subsubsection{护城河}
        古城外的护城河开发利用不足,作为古代城防工程的护城河是古城城市结构中颇具特色的构成要素之一, 对城市形态和空间结构的形成都有重要影响。如今护城河的在军事防御方面的传统功能不复存在,但应当合理开发护城河,使其拥有现代化的价值。
        \begin{figure}[H]
            \centering
            \includegraphics[width=3cm]{图片 29.png}
            \caption[plain]{干涸的护城河\\资料来源:团队摄于平遥古城护城河}
        \end{figure}
        \subsubsection{基础设施}
\textbf{(1).卫生设施:}首先厕所的安排暂不合理,明清一条街上没有设立公共厕所,而在居民区和较为偏僻的小巷,公共厕所之间相差50米左右。此外,厕所的新旧程度不一,有些建设风格与古城整体不符,同时个别厕所内蹲位的设置以及卫生情况不太理想。
\begin{figure}[H]
    \centering
    \includegraphics[width=4cm]{图片 1.png}
    \includegraphics[width=3cm]{图片 3.png}
    \caption{平遥古城内的卫生设施\\}
    \label{fig:my_label}
\end{figure}
\textbf{(2).道路设施:}据考察发现,步行街一直存在不断修路铺路的工程。由于存在道路障碍,一些游客的绕道而行一定程度上减少了部分商铺的经营收入。另外,地面排水设施存在隐患,井盖之下,充满臭气的脏水位线过高,下雨之季积水上涨极大可能造成道路行驶不便。

\textbf{(3).交通设施:}步行街设置的多处栏杆,虽然直接避免了机动车的通行,但是也阻隔了婴儿推车、小型电动车、游览车以及自行车的正常进出,带来不便。此外,在停车场建设方面,北门和南门共有2个停车位,较为合理。但是古城内机动车的摆放仍要加强管理。
\begin{figure}[H]
    \centering
    \includegraphics[width=3cm]{图片 10.png}
    \caption[plain]{栏杆\\资料来源:团队摄于平遥古城内南大街}
    \label{fig:my_label}
\end{figure}

\textbf{(4).标识系统:}通过调查问卷得知,游客对古城内景点的游览图和指示牌满意程度不高,指示牌不够亮眼,存在数目不足的问题。
\begin{figure}[H]
    \centering
    \includegraphics[width=8cm]{标识.jpg}
    \caption[plain]{平遥古城内的标识系统\\资料来源:团队在2021年3月摄于平遥古城南门}
    \label{fig:my_label}
\end{figure}
你看人家那个丽江古城、凤凰古城,灯火辉煌的多好看。本来工程量不大,现在动工的影响挺大的。
\begin{flushright}
——砖雕店老板,2021年3月
\end{flushright}
\textbf{(5).灯光设施:}夜晚的古城,光源大部分来源于商铺,而用于道路照明的基础设施,明显不能够发挥作用,中心的市楼在晚上不被照亮,亮化设施存在开发潜力。
\begin{figure}[H]
    \centering
    \includegraphics[width=6cm]{市楼亮化.jpg}
    \caption[plain]{晚上昏暗的市楼\\资料来源:团队在2021年3月摄于平遥古城南门}
    \label{fig:my_label}
\end{figure}
\textbf{(6).消防设施:}古城内放置的消防设备较少,只有城墙及部分景点有消防沙及水缸的放置。但是在居民区及古城街道等地方,消防设备放置较少且不规范。对木质建筑较多的古城存在较大威胁。
\begin{figure}[H]
    \centering
    \includegraphics[width=6cm]{图片 15.png}
    \includegraphics[width=3cm]{图片 17.png}
    \caption{消防设施}
    \label{fig:my_label}
\end{figure}
\textbf{(7).环卫设施:}古城内居民垃圾分类意识不强,没有关于垃圾分类的规章制度。而乱七八糟的垃圾随意混杂在一起会对其他可回收利用的垃圾造成二次污染,导致可回收利用的垃圾不可再回收,使得资源浪费。

    \subsection{平遥古城建筑保护利用现状}

    古城内的大部分建筑都为砖、木制结构,大部分古建筑都有不同程度的破损。建筑主体的破损直接影响到古建筑的完整性和游客的体验感。景点内的水缸有掉漆现象,双林寺的大门也比较破败,直接影响到游客对景区的第一印象。
    \begin{figure}[H]
    \centering
    \includegraphics[width=3cm]{图片 19.png}
    \includegraphics[width=3cm]{图片 20.png}
    \includegraphics[width=3cm]{图片 21.png}
    \caption{古建筑}
    \label{fig:my_label}
    \end{figure}
    古城内城墙破损现象比较明显,北门和南门附近的城墙都有较严重的破损。城墙损坏严重,采用的填土护坡方法,这种方法采用在古城墙一侧堆砌土方,用于加固原有城墙使之免受自然力侵蚀,此种方法虽然可以最大程度上还原古城面貌,但是防护效果较差,动用土方较多,且在后期维护时难以划清界线,为今后的养护工作带来不便。且相较用石头堆砌, 土方结构较为危险。
    \begin{figure}[H]
    \centering
    \includegraphics[width=5cm]{图片 23.png}
    \includegraphics[width=5cm]{图片 24.png}
    \caption{破损的城墙}
    \label{fig:my_label}
    \end{figure}
    \subsection{平遥古城非物质文化遗产保护利用现状}
我们这个面馆还是有一些老顾客的,让顾客写下这些便利贴,有对我们的建议与自己的心愿。我们是想通过这个监督自己,更好的满足顾客的需要。

\begin{flushright}
    ——北门老刘家罐罐面店主老板娘,2021年3月
\end{flushright}

从北京回来以后,我们更多的是想在家乡发展,平遥古城以晋商文化闻名世界,我们想把自己家乡的这种文化传承下去。经营的过程中,利益肯定是一方面要考虑的因素,当然我们更多的是要为顾客考虑,不能单纯的是为了盈利,主要是有这个情怀在里面。
\begin{flushright}
    ——洪武记饭店青年创业者老板,2021年3月
\end{flushright}
\hspace*{\fill}

继承是发展的前提,发展是继承的必然要求。

垂暮蔼蔼,引水活源。对世界文化遗产的保护不仅要让故有之姿恢复原貌,同时也要将古城融入现代社会的潮流中。原封不动的维护只会将这座历史的遗迹束之高阁,进而曲高和寡。要让古城洗尽尘埃,带着历史的气息贴近群众,必然要把旅游业作为古城保护与发展的重要战略。

在过去几十年的发展中,古色古香的建筑城楼无疑成为平遥古城一张亮丽的名片。古城以极其完整的姿态与较好保留的明清时期县城的基本风貌入选世界遗产名录,但在实际发展的过程中,古城所蕴含的非物质文化遗产在一定程度上并没有得到很好的传承。

        \subsubsection{民俗传承}
传统的地秧歌,流星锤,背棍等活动平日里较为少见,而是成为了一些节日的独特产品。价格高昂的场景剧又见平遥让饶有兴致的游客望而却步,民俗文化的流失削减了古城内该有的丰富灵动的生活气息
        \subsubsection{品牌内涵}
推光漆器背后那些传统工匠对它一成不变的坚守和精益求精的态度没有被发扬出来,大师的字画和笔墨并没有打开与它匹配的足够的市场,, 从游客角度,针对低年龄段的文创产品寥寥无几,例如童年时代的纸鸢和陀螺,对亲子游的群体吸引力不强。
        \subsubsection{晋商文化}
在疫情的冲击下,大多数商铺全部关闭,熙熙攘攘的步行街立刻变得无人问津, 当店主和员工失去了唯一的经济来源而陷入困境,“穷则思变,艰苦创业,开拓进取”晋商精神,以及诚实守信的儒家思想,将如何发扬也成为了一个亟待深思的问题。
        \subsubsection{风土人情}
纵观商业集中地区——明清一条街,各种类型的商铺都有存在,平遥古城特色商铺混杂经营于众多商店中,明清古街更像是普通的商贸区。古城内与古城外售卖种类相差无几。

另外户籍居民中的部分群体在非户籍人口进城经商挤占市场份额的冲击下迁往外城,原生居民也带走一部分古城本土文化的气息。2020 年新冠疫情加速了这一进程,随着古城整顿与修复的进程,本地居民的收入水平也不足以支持原有生活。外来居民在一定程度也冲淡了本土文化。
        \subsubsection{文化产品}
古城内一些文创商店所售皆为日常生活所见的物品,稍加一些古风、古玩的特点,且物价高,导致不仅销售量极低,同时无法满足游客需求。平遥古城传统手工技艺闻名世界,许多慕名前来的游客见到的更多的是老店——推光漆器、老陈醋、平遥牛肉等。尽管价格对不同的群体反响不同,但是产品种类少、价值不副其实是游客共同的反映。
\begin{figure}[H]
    \centering
    \includegraphics[width=8cm]{文创1.jpeg}
    \includegraphics[width=3cm]{文创拼图.jpg}
    \caption[plain]{平遥古城内的文创产品\\资料来源:团队在2021年3月摄于平遥古城内文创店(左一、二)\\右一来源于平遥古城景区官方服务平台公众号}
    \label{fig:my_label}
\end{figure}
同时,产品同质化严重,导致游客对官方与非官方产品的辨识度不高,消费环境安全系数低。商铺所售商品仅以产品命名,缺少品牌意识。
\begin{figure}[H]
    \centering
    \includegraphics[width=8cm]{游客目的.png}
    \caption{游客目的饼图}
    \label{fig:my_label}
\end{figure}
从游客角度来看,做客平遥古城旅游,有很大一部分原因是为了解风土人情,晋中文化。事实上,游客在人文古迹旅游的实质是以相对面积小的景点为依托,接受一种别样的生活环境 ,从而代替体验整个地域。

古城悠久的人文历史确实可以吸引四方游客来此地进行一场“古今对话”的旅行,但是要以此作为旅游业发展的全部基点是远远不够的。短暂的接受历史知识、接触独特的晋商文化无法让游客真正体验到一次平遥古城之行的乐趣。必须从多个反面考虑,结合“个性旅游”、“定制旅游”的新趋势,基于古城自身的特色,为游客的旅程增添新的涵义与内容。

从旅游的六大因素——“吃、住、行、游、购、娱”分析,游客在古城的最大消费主要集中在饭店、酒店、商铺以及娱乐性活动的参与。为了能让游客在旅行的过程中留下深刻印象,毫无疑问要把平遥古城传统的、个性的方面挖掘出来,将平遥本地特色与游客集中点联系起来,以游客需求为导向,变作古城的一种品牌。

当然,从古城保护的方面将,我们不仅要保护古城的“存在”,也要保护古城的“意识”。
\\\hspace{\fill}\\

(疫情)以前还有些游客,现在几乎没有人了,就是凑合的过日子,收入就没有来源了。

\begin{flushright}
    ——五则副食店老板娘,2021年3月
\end{flushright}

古城交通管控的比较严,我们这样的三轮车进不去,人家大门店也不招我们这些人,想挣点小买卖也得辛苦点出去走街串巷。

\begin{flushright}
    ——在北门和西城村贩卖糖葫芦的街头商人老奶奶,2021年3月
\end{flushright}

(对自己的生活)非常满意。国家不管农民还是市民都是补贴 123 块,全县统一的。

\begin{flushright}
    ——街边闲坐的大爷,2021年3月
\end{flushright}

不行,不够活,你们看我这衣服——穿了好几年了。我们出来没有买过古城里的东西,也不知道贵不贵。封城的时候,是我们家亲戚给我们捎进来一些蔬菜。
\begin{flushright}
    ——西城村老奶奶(市民),2021年3月
\end{flushright}

生活拮据,贫苦落后,衣服穿着较为贫穷,冬天燃煤取暖,受亲戚接济一些生活用品;因为文化水平低,身体年迈,没有挣钱渠道;古城居民贫富差距太大。
    \subsection{平遥古城景区综合治理现状}
    从实地采访的结果来看,政府在发展平遥古城的经济的过程中,实施了诸多有效措施, 基本改变了古城贫穷落后的面貌,居民的幸福感指数大幅度上涨。但是随着旅游业的发展, 其自身暴露的许多问题也同样充斥着居民的生活,从另一方面给古城发展带来不利影响。

        \subsubsection{居民区风貌落后,发展水平较低}
    \begin{figure}[H]
    \centering
    \includegraphics[width=4.5cm]{建筑.jpg}
    \caption[plain]{平遥古城内的旧建筑\\资料来源:团队在2021年3月摄于平遥古城西城村}
    \label{fig:my_label}
    \end{figure}
    旅游业的发展一定程度上拉大了居民贫富差距,新冠疫情的爆发使这一问题彻底暴露出来。居民区出“空房无人”的情况,部分居民生活困难,相对封闭的环境导致思想趋于保守,居民区逐渐“空心村化”。

    由于完整的文物古迹多数位于古城南部,北部游客量极大程度上少于南部。客流量的不平衡导致了南北租金差距大,加上北部大面积为原始村落,属于本地居民生活区,古城北部的发展明显不如南部。

    此外,由于土地征收,部分农民转农从商,同时由于文化水平不高,只能从事一些服务型产业,或者租用个人门店从事小型商业,经济收入依附于旅游业,极具脆弱性,因此遭受疫情的冲击影响较大。窘迫的经济环境迫使年轻劳动群体虽然仍从事古城旅游业,但选择离开古城生活。对于一些行动不便,缺乏知识的老年群体只能依靠国家补助维持生活。

    再者,平遥古城属于世界级文化遗产,客源地不乏为经济水平极高的区域。平遥古城所展现出的落后农村气息会在视觉感官上影响游客旅行体验。有必要改变整体村落风貌,进一步改善居民生活水平。

        \subsubsection{旅游业体系单一,呈畸形发展趋势}

    随着古城旅游业的发展以及旅游市场的进一步开放,市场经济的盲目性也日益凸显,诸多人员投入到服务行业,导致旅游业所提供的服务质量参差不齐,甚至影响到游客利益。“黑店”、“宰客”的不良现象不断发生,欺诈型消费直接导致游客对古城的印象与满意程度下降,继而影响古城旅游业的整体发展。

    另一方面,古城旅游业主要依靠门票来拉动。官方价格为125元的古城通票,虽然包含22个景点,但是普遍受到游客和商铺的非议。平遥古城“门票经济”特征突出。

    此外,古城占地面积 2.25 平方千米,城内街道错综复杂,导游与游览车服务占据重要作用,成为游客游玩必不可少的部分。从事此行业的人群众多,有当地旅行社集中管理的专业导游,也有本地居民为谋生借助自身的熟悉度为游客导路,比较特殊的是义务导游。这类导游分为两类群体,一部分是当地小学组织的学生义务讲解,这类群体虽然专业化水平低, 经验欠缺,但是深受游客喜爱,受众度比较高。另一部分是政府管理的旅游景点,委派专门员工负责为到访游客进行讲解,但是此类景区多是知名度较低,并且雇佣的员工也大多为非户籍群体。从种类上说,导游行业讲解人员各具特色,可以带给游客不同的体验。但是从数量上说,义务导游最为少见,居民导游次之。而占比比较大的旅行社导游与游客接触最多,管理问题也最大。根据实地考察发现,导游带团会到与旅行社有合作关系的固定商铺,这对一些中小型商铺的影响比较大,间接导致客源减少。使得商铺行业呈依托资产发展的体系,游客也从客观上无法接触更多的商品。

    由于旅游业所推产品种类较少,即文化创意产品、特色活动少,古城旅游业收入来源单一。政府要维持古城维修与保护的费用,只能依附于门票价格。然而,从数据分析来看,游客停留时间多为短期,并且个性爱好不同,“打包式门票”颇有些强制消费的意味。

        \subsubsection{防控意识不高,古城内外标准差别大}

    疫情期间,受封城的影响,居民物资补助困难,行政管理松懈。有一重要实例是,平遥古城景区管理有限公司于 2021 年 2 月 24 日下发文件《关于平遥古城景区调整疫情防控措施、3D 灯光秀恢复演出通告》,但在三月末景区入口处的防疫设施已经停用,工作人员对进城游客的检查较为松弛。见图
\begin{figure}[H]
    \centering
    \includegraphics[width=4cm]{古城南门.jpg}
    \includegraphics[width=6cm]{洗脸盆.jpg}
    \caption[plain]{古城南门入口\\资料来源:团队在2021年3月摄于平遥古城南门入口}
    \label{fig:my_label}
\end{figure}
同时,古城内外卫生标准不统一,在平遥古城外围区,公共基础设施维护弱于城内。过渡区发展滞后,城墙内外景象对比强烈。

一方面是古城受旅游业的影响飞速发展,商业经济不断扩大,另一方面是政府的政策与建设步伐较慢,与时代和发展的协同性低。

        \subsubsection{景点规划协同性差,旅游路线联通性低}
\begin{figure}[H]
    \centering
    \includegraphics[width=6cm]{地图.png}
    \caption[plain]{平遥古城鸟瞰图\\资料来源:Google地图}
    \label{fig:my_label}
\end{figure}
平遥古城的景点较为分散,城外著名景点——双林寺和镇国寺分别距离古城南门 8 公里、15 公里左右,且在古城东西分布局,与火车站、高铁站距离同样很远,对于占比较大的一日游游客来说,分散的景点很难形成一个有效的旅游方案,在辗转的路上消耗就极大。对于古城内的景点同样如此,大部分景点东西对称分布,而景点之间的线路存在单一、游览价值低的情况,例如要游览日昇昌旧址与华北第一镖局博物馆就要将西大街来回走两遍, 倘若要观看有定点时间表演的景点,路程更是冗长。

景点的分散造成了古城周围的空白区较多,也许一些地方在游客游览的过程中会经过, 但是缺少相应的游览点,游客的消费只能集中在来往的路途上。因此,鉴于游客量在不同地方的出入,政府对不同地方的管理程度也就不尽相同。
\section{平遥古城游客体验评价}
    \subsection{评价指标的选取}
    游客评价最初源于 商业市场营销中的消费者满意度调查。\upcite{李智虎2003谈旅游景区游客服务满意度的提升}从管理学和营销学的视角, 游客体验就是游客 的满意度,\upcite{余向洋2006游客体验及其研究方法述评}
    之后逐渐被用于旅游学术研究中。游客在完成游览过程后, 对游程中各方面的满意度, 不仅会产生“ 晕轮效应”\footnote{晕轮效应(有时称为晕轮误差)是一个人,公司,品牌或产品在一个区域中产生积极印象,从而积极影响他人在其他方面的观点或感觉的趋势。\upcite{nisbett1977halo}} 而 影响到旅游目的地的整体形象, 还会通过经历描述 和体验分享, 影响客源地潜在消费群体的主观感知 和旅游消费决策。
    我们采取问卷调查的方式,共设计了三份问卷,即针对平遥古城内的居民、游客和商铺设计了不同的问卷。
    \subsection{研究数据及计算}
        \subsubsection{基本数据}
            2021年3月,我们进行了为期一周的调查,共随机发放396份问卷,为保证调查结果的有效性,采用面对面的形式发放问卷,现场填写、现场回收。
            共回收有效问卷320份,有效率为80.8\%,其中游客问卷发放226张,有效回收220张,有效率为97.3\%,调查样本总体情况见表2
\begin{table}[H]
\centering
\caption{平遥古城景区调查游客的基本属性}
\begin{tabular}{p{2.5cm}p{3cm}p{2cm}p{2cm}}
    \toprule
    项目 & 属性 & 样本数 & 百分比(\%)\\ 
    \midrule
    性别&男 &96 & 42.4 \\
    & 女&130 &  57.6\\
    \cmidrule{1-4}
    年龄& 18岁以下& 44 &19.5  \\
    &18-25岁&50 & 22.1  \\
    &26-45岁&112 & 49.5  \\
    &46岁以上&18 &  7.9 \\
    \cmidrule{1-4}
    受教育程度& 小学及以下&42 &18.9 \\
    &初中/职业高中& 18&8.1   \\
    &普通高中/专科&44& 19.8 \\
    &普通本科及以上&118 & 53.2  \\
    \cmidrule{1-4}
    职业&学生&68&31.5\\
    &事业单位&24&11.1\\
    &企业单位&68&31.5\\
    &党政机关&12&5.6\\
    &自由职业&24&11.1\\
    &其它&20&9.3\\
    \cmidrule{1-4}
    客源地&东北&22&9.6\\
    &华东&50&21.7\\
    &华北&136&59.1\\
    &华中&8&3.5\\
    &华南&12&5.2\\
    &西南&0&0\\
    &西北&2&0.9\\
    \cmidrule{1-4}
    出游方式&背包客&52&33.8\\
    &自行游&52&33.8\\
    &跟随旅行团&2&1.3\\
    &和朋友家人组团游&26&16.9\\
    &其它&22&14.3\\
    \bottomrule
\end{tabular}
\end{table}
在被调查的220位游客中,女性游客所占比例较大,占被调查对象的57.6\%,女性高于男性15.2\%,被调查的年龄层次主要是26-45岁,占49.5\%,其次是18-25岁和18岁以下的游客,分别占22.1\%和19.5\%,在来访的游客中,占最多的是学生和企业单位人员,各占31.5\%,其次是事业单位人员和自由职业者,分占11.1\%,受访游客的学历中,最多的是普通本科及以上,占53.2\%,之后是普通高中或专科,占19.8\%,再次是小学及以下,占18.9\%,最后是初中或职业高中,占8.1\%,受访游客多来自华北地区,占59.1\%,其次是华东和东北,分占21.7\%和9.6\%,其余地区总和不超过十分之一,可以看出平遥古城的游客中北方人居多,尤其是华北地区,客源地较为狭窄。
\subsubsection{游客的活动特征}
\textbf{(1).出游方式:}受访游客中,背包客和自行游的游客最多,各占33.8\%其余多数以和朋友家人组团为多数,跟随旅行团的游客最少,只有1.3\%。

\textbf{(2).旅游目的:}问卷显示,遥古城的游客的旅游目的大多了解风土人情,晋中文化,占到总数的35.2\%,其次是和朋友玩,培养感情,占22.9\%,再次是疫情之后的放松,缓解压力,占20\%。

\textbf{(3).游客停留天数:}一半左右(50.4\%)的游客选择在平遥古城停留2-3天,其余38.9\%的游客选择只停留一天,停留4-7天的游客数量和7天以上的游客数量大致相同,都占5.3\%。

\textbf{(4).住宿选择:}其中过夜的游客在古城的住宿方式大多选择民宿(61.5\%),其余28.8\%的游客选择酒店住宿,剩余游客选择住在亲戚朋友家9.6\%。

\textbf{(5).购物体验:}对于平遥古城内景区的商品,认为物美价廉的游客最多,占33\%,还有31\%的游客认为物价偏贵,还有25\%的游客认为商品质量参差不齐,其余11\%的游客认为商品一般,不满意。可以看出平遥古城相较其他景区,物价普遍上还是要低一些。

\textbf{(6).古装意愿:}在古城中身着古装是一种新兴的旅游体验,如果在政府鼓励的情况下,有35.8\%的游客表示一定会穿着古装出行,60.4\%的游客认为随意,可以看出古装爱好者的群体还是占有相当的比例的,可以借此发展更具沉浸式的旅游体验。
\subsubsection{构建评价因子集}
问卷要求游客采取五点量表\footnote{五点量表的评语等级分为非常满意、满意、一般、不满意、非常不满意等 5个等级。}法进行打分。分别是1-5分
在用综合评价法考察游客的主观感受时, 选取景观完整度、真实性、建筑、风土人情、特色住宿、特色餐饮、导游服务、标识系统、卫生整洁程度、交通便利程度、商业化程度、购物品12个因子,构成评价因子。
\subsubsection{单项因子打分}
让游客对所在的景区的 12个因子分别下一个 评语, 评语集为
\[V=
\begin{bmatrix}
\text{非常满意}(v_1)&\text{满意}(v_2)&\text{非常满意}(v_3)&\text{不满意}(v_4)&\text{非常不满意}(v_5)
\end{bmatrix}
\]

从因素集到评语 集建立模糊关系, 算出各自的隶属度。例如调查结 果表明, 对于景观完整度而言, 27.2\%的游客表示“ 很 满意”, 52.2\%的游客表示“满意”, 17.4\%的游客表 示“一般”, 0 \%的游客表示“不满意”, 3.3\%的游客 表示“很不满意”, 于是可对景观完整度因子得到如下 模糊评价向量:
$\begin{bmatrix}
0.272&0.522&0.174&0&0.033
\end{bmatrix}$
我们可以将这样的向量列成表,计算每种因子的平均分。
\begin{table}[H]
\centering
\caption{平遥古城游客满意度统计}
\begin{tabular}{cccccccc}
    \toprule
    因子&非常不满意&不满意&一般&满意&非常满意&平均分&排名\\
    \midrule
    景观完整度&0.033&0	&0.174&0.522	&0.272&4.003  &4\\
    真实性&0.022	&0.011	&0.129&	0.581	&0.258&4.045&2\\
    建筑&0.034	&0	&0.09	&0.528& 0.348&4.156&1\\
    风土人情&0.034	&0.011	&0.205	&0.5	&0.25&3.921&5\\
    特色住 宿&0.023	&0&	0.218&	0.46&	0.299&4.012&3\\
    特色餐饮&0.022	&0.033&	0.242	&0.451	&0.253&3.883&7\\
    导游服务&0.046	&0.011	&0.287&0.402	&0.253&3.802&9\\
    标识系统&0.056	&0.045	&0.191	&0.461	&0.247&3.798&10\\
    卫生整洁程度&0.034	&0.045	&0.193	&0.432	&0.295&3.906&6\\
    交通便利程度&0.034	&0.068	&0.17	&0.455&	0.273&3.865&8\\
    商业化程度&0.069&0.069&0.345&0.287&0.23&3.54&12\\
    购物品&0.056&	0.056	&0.315	&0.36	&0.213&3.618&11\\
    \bottomrule
\end{tabular}
\end{table}
\subsubsection{分析与结论}
可以看到,平均分分最高的是建筑方面,最低的是商业化程度和购物品,这是一个旅游景区的共性问题,也是经济学上的基本供求关系决定的,我们在此不展开讨论。排在第十名的是标识系统,之后依次是导游服务,交通便利程度和特色餐饮,这说明平遥古城的标识系统,交通,和餐饮是游客不满意的三个主要问题,但标识和导游对游客体验的重要性相对并不高,基于综合性考量,我们认为影响平遥古城游客体验的三个主要短板是交通,卫生,和标识系统。

\section{平遥古城文化遗产韧性的提升策略}
    \subsection{环境优化的对策}
        \subsubsection{设施的管理和增设}
\textbf{(1).落实管理,真抓实干:}负责管理古城的会管人员应及时加强管理,把管理落到实处,真抓实干,把厕所建设好,把卫生提上去,增加相应的居民设施,提高工程队作业效率,并在施工时采取降噪和降尘措施。

\textbf{(2).安装智能栏杆,加强宣传力度:}对于交通设施来看,可以匹配上智能路障或者栏杆,感应到大车经过时不予通过,而类似婴儿推车的也则可以以予通过,同时,也要做大宣传加强引导,提醒游客古城内不能进入大车,机动车的摆放统一要求。

\textbf{(3).完善标识系统:}采取游客反馈的意见可以在关键路口合理建设一些指示牌和引导图,在火车站等人流量大的地方加强宣传,引导游客前来游玩,从而促进古城经济发展。

\textbf{(4).增设消防设施,提高居民消防安全意识:}在古城居民区及道路上定点放置消防设施,并且定期对灭火器等消防用具进行功能检测和维护。同时,在古城内开展相关消防宣讲及演练活动,加强古城内居民的消防安全意识。

        \subsubsection{采用立交式排水系统改善古城排水问题}
        路面采用集中排水方式, 即:适当抬高路缘石, 每隔一定距离设置一道泄水槽, 路面积水集中流入泄水槽, 最终排出立交区。

        在路基边坡台面适当高度建设第一道平台, 以阶梯形式设置第二道边坡平台, 最终建构起边坡排水系统。同时,为了减少上一级边坡坡面积水对下级边坡产生冲刷, 在每道平台上,应设置相应阻拦机制。
        \begin{figure}[H]
    \centering
    \includegraphics[width=8cm]{排水.png}
    \caption[plain]{立交式排水系统\\图片来源文献\cite{魏明祥2002丹东古城互通式立交的排水设计}}
    \label{fig:my_label}
        \end{figure}
        毗邻地带地表水指路段外自然坡面积水。具体措施是在边坡坡顶外设置截水沟, 沟底建设急流槽, 自然坡面上的流水流入截水沟, 再经急流槽流入边沟或直接经截水沟排出立交区。\upcite{赵阔宇2011浅谈边坡生态防护技术在古城墙保护中绿化的应用}
        \subsubsection{解决供暖和秸秆燃烧的问题}
        首先要充分发挥新闻媒体的舆论引导和监督职能,利用、条幅、广播、电视、手机微信、手册等方式开展秸秆禁烧宣传教育工作,增强广大农民群众对秸秆焚烧处罚措施和影响危害的认识,并大力宣传秸秆综合利用的奖励政策,引导农民朋友自觉树立、“不想烧”、“不能烧”、“不敢烧”的意识。

        通过加大科技应用以及对公众的培训力度,按照肥料化、饲料化、基料化、能源化和原料化“五化”技术路线,有效拓展秸秆增值新通道。综合规划、合理布局,因地制宜开展秸秆资源化利用,稳步提高秸秆综合利用率,真正让开展秸秆综合利用的农民有利可图,有钱可赚。
        \subsubsection{有针对性地加强古城内的绿化工程}
        对于古城墙的维护工作可以采用制草护坡防护的方法,方法如下:在由于年久失修的小部分城墙形成小型的边坡上,种植抗性好外观好且耐虫害的植物,该方法不仅可以对古城墙起到较好的保护作用,并且最大限度减少对于古城原生环境的破坏。

        古城内部道路空间有限,我们提议设置街头装饰型绿化,以立体绿化等形式提升街道的绿化覆盖, 改造发展复层群落的种植带式绿化, 从而有效增加绿地面积和绿化率。
        
        在古城外的工业区应设置防护带,工业区与古城间的绿化带宽度不得少于50米,绿化树种应选择对工业有害物质抗性强的乡土树种。\upcite{赵阔宇2011浅谈边坡生态防护技术在古城墙保护中绿化的应用}
        \subsubsection{施工污染的解决措施}
\textbf{(1).噪音污染的解决措施:}\\
a.发挥环保部门的管理和监督作用

古城相关环保部门可以根据当地人民政府批准的《噪声功能区域划分》加强监督,并对噪音源相关情况进行报备。同时限制道路施工时间,减少对周围居民及游客的影响。\\
b.动员鼓励“公众参与”的监督办法

古城当地居民在受到施工造成的噪音污染时,有权对造成环境噪声污染的单位和个人有权进行检举和控告,保护自己应享有的环境权益。而且公众还有权监督环保部门执法人员的行政行为,促使环保部门执法人员必须按照环保法律法规秉公执法、不拘私情,以保证施工噪声污染防治得到建设单位有效的治理。\\
c.从源头上防治

在古城内施工时尽量使用低噪声的施工机械和措施减少噪音污染的产生。调整施工时间,避免在居民休息时间及游客量较大的时间施工,减少噪音污染的影响。\upcite{林哲宇2017浅析建筑施工中的施工环境污染问题与防治策略}

\textbf{(2).古城施工产生污染环境的粉尘和垃圾整治解决的对策:}

在古城建筑工程施工过程中,按时进行洒水工作减少粉尘的污染。施工现场须按标准进行物料堆放,做到整齐、有序、有标识。在道路施工中减少在巷子中的物料堆积,及时将巷中的物料清理,减少对居民及游客出行的影响。

\textbf{(3).古城施工污水的治理措施和解决的对策:}

施工准备阶段做好排水系统的疏通工作,确保施工污水和生活污水能及时的排人城市污水管网。施工污水需经过污水处理确保对环境无污染情况下再进行排放
        \subsubsection{从管理制度进行改善}
管理部门应将古城各建筑的情况记录在案,定期进行建筑的维护和修缮工作。在不破坏古城原有风格的前提下,利用现代化的科技手段对古城建筑进行修复,使得古城具有“新”“旧”结合的健康面貌。

    \subsection{非物质文化遗产的活化利用}
        \subsubsection{开展各种形式的文化活动,增强游客的参加度}
防控疫情不等于一刀切,停止举办所有的文化节日活动。我们需要在建立一套疫情防控常态化的机制下,逐渐举办多种类型的文化节日,让非物质文化的展现拥有更大更多的平台,让文化活起来。像每年举办的中国年,国际摄影大展为平遥古城的知名度和经济收入带来了不错的影响。
        \subsubsection{传承传统晋商精神,将情怀为商业文化的核心}
大力发扬晋商文化,运用到疫情常态化的旅游发展上。将晋商文化中孕育的博大宽厚的胸怀,兼容并蓄的经营气度以及求同存异的经营策略,运用到古城商铺的发展里去。例如创新云旅游的方式,在收到冲击时保持一种情怀大于利益的心态。

开展晋商文化活动,营造特定氛围。商铺可以通过培训员工,在对游客的接待中再现明清晋商精神。政府也可以举办特定的晋商实景剧活动,让游客体验晋商的生活,感受晋商文化,学习票号技术。

        \subsubsection{推动“非遗”创造性转化、创新性发展,丰富游客体验}

如果将游客消费点比作一件件商品,而古城的非物质文化遗产就是这些商品精美的包装。推动“非遗 + 住宿”、“非遗 + 创新”、“非遗 + 互联网”等快速皆合,以游客喜闻悦见地方式推广开来,打造深受年轻人喜爱的项目和作品。

        \subsubsection{打造特色古城氛围,营造古城历史气息}
\begin{figure}[H]
    \centering
    \includegraphics[width=4cm]{汉服.jpeg}
    \includegraphics[width=7cm]{表演.jpg}
    \caption{平遥古城内身着古装汉服的游客(左)来源:团队于2021年3月摄于平遥某客栈\\平遥古城内的文化表演(右)来源:团队摄于平遥古城南大街街口}
    \label{fig:my_label}
\end{figure}
不可否认的是,平遥古城最大的特色体现在它完整度极高的“古”上,“古”在城墙建筑,“古”在风土人情。

近年来随着弘扬传统优秀文化意识的不断加强,“古风”文化在青年群体里正带一股来清新的潮流。“着一裳翩翩华服,走一遭古人之路”,年轻群体以这样一种方式打卡人文古迹的现象在古城内非常常见。对于古城内逐渐庞大的这类新群体,政府可予以支持和鼓励,组织兴趣爱好者成立相关协会,同时宣传本地的汉服商铺,推动汉服租用,摄影门店的协同发展,打造“一条龙”服务。

改造传统破旧村落,建设和古城相符的明清建筑风格,同时在古城北部开设更多的非物质遗产商铺,或是连锁本城老字号,以此获取游客注意力,增强旅行体验。
        \subsubsection{创造古城文物衍生品,形成特色古城品牌}
采取线上推广的形式,设计古城独属的文化形象或者代言人,征集文物的艺术形态,作为古城文化创意产品的品牌标识。同时可以和新媒体公司合作,创造品质高、口碑好的纪录片、综艺片或者影视作品。
\begin{figure}[H]
    \centering
    \includegraphics[width=4cm]{表情包.jpg}
    \includegraphics[width=8cm]{古城小镖师.jpg}
    \caption[plain]{带有平遥古城元素的表情包(左)平遥古城的创意动画《古城小镖师》(右)\\资料来源:左图来源于团队自摄加工,右图来源于平遥古城景区官方服务平台公众号}
    \label{fig:my_label}
\end{figure}
    \subsection{制度设计提升}
(1).针对劳动群体的不同情况,政府应该安排相关非遗传承人培训技艺技巧。在传承非物质文化遗产同时,通过推广技术的手段来促进非遗特色产品的增进,以“扶智就业”的方式带动经济的发展。例如引进人才,开设传统技艺班,从而建立一套完整统一的保护机制,而后期衍生的产业链既可以传承文化,也可以带动附近居民就业,增加居民经济收入。


(2).形成独特且系统的旅游体系,首先需制定严格的旅游市场管控条例,严厉打击不法经营的现象。其次可通过制定官方标识,将古城内的商铺划分级别,集中管理。同时,设立专职部门监督导游行业。例如解决游客纠纷,只有在官方平台已挂牌的导游方可为其服务。


(3).传承传统文化,每一代都有着不可推卸的责任。借鉴平遥小学义务讲解的事例,可以将平遥县乃至山西省高校列入志愿讲解的行列,以计算志愿讲解时长的方式,由学校组织并鼓励学生前往平遥古城学习讲解。


(4).基于门票方面,可实行“套餐”销售,景区根据不同游客群体的消费心理,可设计多种不同景点结合、门票优惠的方案,增强吸引力。


(5).针对工程项目改造,政府可以借鉴古今中外旅游景点安全事故防范措施,增强对工作人员的考核监督。有计划的推进工程措施,不急功近利,兼顾民生与经济。


(6).开发古城小巷路道,增强景点之间的联通性。同时规划出简洁周全的游览路线,结合景区门票套餐方案的设计,使游客的选择更加实惠个性。
\section{结语}

平遥古城不仅对研究中国历史,借鉴于当今具有不可估量的重要价值,而且作为世界遗产在世界文化宝库中占据着重要地位。在疫情防控期间,平遥古城采取了一系列的措施应对疫情,刺激旅游恢复。其中不乏可圈可点的方面,如政府发令减免商铺租金,疫情防控后期实行免门票的措施等。然而,在如何增强平遥古城对当地经济社会恢复发展的韧性拉动作用方面,仍存在很大的挖掘空间。平遥古城开发中的保护和发展方面,例如古城墙和基础设施的保护,以及文化产品的创新都应纳入考虑范畴之内。疫情之下,衷心希望我们的这些建议能够被关注到并加以采纳,对疫情下的平遥古城开发和可持续发展有所帮助。祝愿平遥古城能够在各个方面做到尽善尽美,成为一处人人流连忘返的旅游胜地,大大提升游客体验感。

本文在实地调研的基础上,获取了古城发展情况以及疫情过后的实际情形。伴随着古城旅游业资源的不断挖掘,现实发展的矛盾也接连涌现,疫情的爆发打破了古城保护和发展勉强维持的平衡。政府、游客、商铺之间的利益冲突从另一方面反映了古城保护与利用的侧重。政府也在游客不断的反馈中积极探索新的角度与方向,在保护的前提下合理利用古城资源。但是“大刀阔斧”与“不切实际”的改革现象仍然存在,新的措施在古城相关群体中的推行同样举步维艰。

经过千百年岁月的轮回,平遥古城为后世留下了宝贵的文化遗产。基于游客体验视角的引导,统筹保护古城文物和发展旅游业是目前古城突破困境的出路所在,如何传承并且使其在现代社会推陈出新将会是一个永恒的话题。
\nocite{*}
\bibliography{pingyao}

\end{document}