\documentclass[UTF8]{ctexart}
\usepackage{graphicx}
\usepackage{float}
\usepackage{ragged2e}
\usepackage{amsthm}
\usepackage{amssymb}
\usepackage{amsmath}
\usepackage{wrapfig}
\begin{document}
\begin{equation*}
    \begin{split}
        \lim_{\Delta x \to 0} \frac{\tan({x+\Delta x})-\tan{x}}{\Delta x}&=\lim_{\Delta x \to 0}\frac{\frac{\sin({x+\Delta x})}{\cos({x+\Delta x})}-\frac{\sin x }{\cos x} }{\Delta x} \\
        &=\lim_{\Delta x \to 0}\frac{1}{\Delta x}\frac{\sin({x+\Delta x})\cos{x}-\cos({x+\Delta x})\sin{x}}{\cos({x+\Delta x})\cos{x}}\\
        &=\lim_{\Delta x \to 0}\frac{\sin{\Delta x}}{\Delta x}\frac{1}{\cos({x+\Delta x})\cos x}\\
        &=\frac{1}{\cos^2 x}=1+\tan^2 x
    \end{split}
\end{equation*}
可导与连续可导的区别\\
\textbf{所谓可导即导数存在},即
\begin{equation}
    f'(x_0)=\lim_{h \to 0}\frac{f(x_0+h)-f(x_0)}{h}
\end{equation}
存在,特殊的,若$x_0=0$,那么即
\begin{equation}
    f'(0)=\lim_{h \to 0}\frac{f(0+h)-f(0)}{h}
\end{equation}
存在,有时这个式子还会写成
\begin{equation}
    f'(0)=\lim_{x \to 0}\frac{f(x)-f(0)}{x}
\end{equation}
(2)和(3)是完全等价的\\
\textbf{所谓连续可导意为导数连续}\\
连续指对一个函数$f(x)$及一点$x_0$来讲,存在等式
\begin{equation}
    \lim_{x \to x_0}f(x)=f(x_0)
\end{equation}

对于一个函数$f(x)$的导函数$f'(x)$来讲,在$x_0$连续意味着
\begin{equation}
    \lim_{x \to x_0}f'(x)=f'(x_0)
\end{equation}
成立,如果一个函数在$x_0$处满足(1)和(5)式,则称为在$x_0$处\textbf{连续可导},若只满足(1),则称为在$x_0$处\textbf{可导}\\
举例来讲,我们了考察一个具体函数
\begin{equation}
    f(x)=\begin{cases}
        x^{\alpha}\sin{\frac{1}{x}},& x>0\\
        0,& x \leqslant 0
    \end{cases}
\end{equation}
若使$f(x)$在0处可导,应满足什么条件?连续可导呢?\\
若$f(x)$在0处可导,那么直接利用(3),有
\begin{equation}
    f'(0)=\lim_{x \to 0}\frac{x^{\alpha}\sin{\frac{1}{x}}-0}{x}=\lim_{x \to 0}x^{\alpha-1}\sin{\frac{1}{x}}
\end{equation}
可导意味着极限(7)要存在,我们知道,这个极限存在的条件是$\alpha>1$(极限存在不包括极限等于无穷大!)\\
连续可导意味着(5),要得到(5)的左半部分的$f'(x)$,首先要求导,我们需要知道在0处的导数,若(7)存在(即$\alpha>1$),经过计算,我们得到(7)式若存在则必为0,于是$f'(0)=0$\\
再对$x>0$的部分求导,就得到了完整的导数
\begin{equation}
    f'(x)=\begin{cases}
        \alpha x^{\alpha-1}\sin{\frac{1}{x}}-x^{\alpha-2}\cos{\frac{1}{x}},& x>0\\
        0,& x \leqslant 0
    \end{cases}
\end{equation}
(5)左半部分的极限意味着左右极限都存在且相等,在0处,其左极限显然是0,我们考察其右极限
\begin{equation}
    \lim_{x \to 0^+}f(x)=\lim_{x \to 0^+}\alpha x^{\alpha-1}\sin{\frac{1}{x}}-x^{\alpha-2}\cos{\frac{1}{x}}=0=\lim_{x \to 0^-}f(x)=f'(0)
\end{equation}
要使该式成立,x的指数不能为负,即$\alpha>2$
\end{document}