\documentclass[hyperref,UTF8]{ctexart}
\usepackage[dvipdfmx]{graphicx}
\usepackage{gbt7714}
\usepackage{float}
\usepackage{ragged2e}
\usepackage{amsthm}
\usepackage{amssymb}
\usepackage{amsmath}
\usepackage{wrapfig}
\usepackage{tikz}
\usetikzlibrary{arrows.meta}
\usepackage{booktabs}
%\usepackage[a4paper,left=3.18cm,right=3.18cm,top=2.54cm,bottom=2.54cm]{geometry}
\usepackage{tabularx}
\usepackage{array}
\usepackage{caption}
\usepackage{hyperref}
\newcommand{\upcite}[1]{\textsuperscript{\textsuperscript{\cite{#1}}}}
\setCJKfamilyfont{song}{SimSun}
\title{SU(2)}
\author{limbo137}
\begin{document}
\maketitle
我们取旋量的一种规范
\[ \lvert \psi \rangle =\begin{pmatrix}
    \cos \frac{\theta}{2}e^{i \frac{\phi}{2}}\\
    \sin \frac{\theta}{2}e^{-i \frac{\phi}{2}}
\end{pmatrix}\]
容易证明其满足
\[\langle \psi\vert \psi \rangle =1\]
为保证上式, $\vert \psi \rangle$的演化方程
\begin{equation}
    \frac{\mathrm{d}}{\mathrm{d}t}\vert \psi \rangle=A\vert \psi \rangle \label{eq:evl} 
\end{equation}

的解有形式
\[\vert \psi(t) \rangle=U\vert \psi(0) \rangle\]
其中 $U$满足 
\[U^\dagger U=1\]
且
\begin{equation}
    \det U =1 \label{eq:det}
\end{equation}

全体这样的算子的集合组成SU(2)lie群

根据演化方程(\ref{eq:evl}),我们有
\[U=e^{At}\]
于是
\[\det U=\det e^{At}=e^{\text{Tr}At}\]
按照(\ref{eq:det}),有
\[\text{Tr} A=0\]
全体A的集合称为lie群SU(2)的lie代数,这里可以看出SU(2)的lie代数都是迹零的反厄米矩阵\footnote{写成矩阵是该群的一种表示}.

我们来看这个群的具体表示,首先我们仍考虑演化方程,我们有时也称其为二维自洽系统
\[\begin{pmatrix}
    \dot{u}\\\dot{v}
\end{pmatrix}=i\begin{pmatrix}
    a& b\\
    c&d
\end{pmatrix}\begin{pmatrix}
    u\\v
\end{pmatrix}\]
中间乘 $i$的目的是让中间的矩阵成为厄米的,解决这种问题的一般思路往往是对角化,而对角化中的矩阵我们使用上面的$U$来做基变换,中间矩阵的特征方程是
\[\lambda^2-(a+d)\lambda+ad-bc=\lambda^2-\text{Tr A}\lambda+\det A=0\]
其本征值是
\[\lambda=\frac{\text{Tr A}}{2} \pm \frac{\sqrt{(\text{Tr A})^2-4\det A}}{2}\]
所以当 $\text{Tr A}=0$时, $\lambda=\pm i\sqrt{\det A}$ 

但我们其实有更好的方式,由厄米和迹零性,我们可以改写上面的矩阵
\[\begin{pmatrix}
    \dot{u}\\\dot{v}
\end{pmatrix}=i\begin{pmatrix}
    z& x+yi\\
    x-yi&-z
\end{pmatrix}\begin{pmatrix}
    u\\v
\end{pmatrix}\]
其中 $a,b,c$是实数,其中 $b,c$定义与上不同,继续改写为 
\[\begin{pmatrix}
    \dot{u}\\\dot{v}
\end{pmatrix}=(ix\begin{pmatrix}
    0& 1\\
    1&0
\end{pmatrix}+iy\begin{pmatrix}
    0& i\\
    -i&0
\end{pmatrix}+iz\begin{pmatrix}
    1& 0\\
    0&-1
\end{pmatrix})\begin{pmatrix}
    u\\v
\end{pmatrix}\]
定义
\[\sigma_x=\begin{pmatrix}
    0& 1\\
    1&0
\end{pmatrix},\sigma_y=\begin{pmatrix}
    0& i\\
    -i&0
\end{pmatrix},\sigma_z=\begin{pmatrix}
    1& 0\\
    0&-1
\end{pmatrix}\]
于是
\[\begin{pmatrix}
    \dot{u}\\\dot{v}
\end{pmatrix}=i\vec{r}\cdot\vec{\sigma}\begin{pmatrix}
    u\\v
\end{pmatrix}\]
且
\[\det (i\vec{r}\cdot\vec{\sigma})=|\vec{r}|^2\]
所以对角化后
\[\begin{pmatrix}
    \dot{u}\\\dot{v}
\end{pmatrix}'=ir\sigma_z\begin{pmatrix}
    u\\v
\end{pmatrix}'\]
其中 $r=\sqrt{|\vec{r}|^2}$ 
\\解是
\[\begin{pmatrix}
    u\\v
\end{pmatrix}'=e^{ir\sigma_z}\begin{pmatrix}
    u'_0\\v'_0
\end{pmatrix}=(I\cos r+i\sigma_z\sin r)\begin{pmatrix}
    u'_0\\v'_0
\end{pmatrix}\]
又
\[U^\dagger irt\sigma_z  U=i\vec{r}\cdot\vec{\sigma}\]
所以
\[\begin{pmatrix}
    u\\v
\end{pmatrix}=e^{i\vec{r}\cdot\vec{\sigma}t}\begin{pmatrix}
    u_0\\v_0
\end{pmatrix}=(I\cos rt+i\vec{n}\cdot\vec{\sigma}\sin rt)\begin{pmatrix}
    u_0\\v_0
\end{pmatrix}\]
其中
\[r\vec{n}=\vec{r}\]
上面的作用相当于以 $\vec{r}$的角速度旋转 

设 $rt=\theta$,有
\[e^{i\theta\vec{n}\cdot\vec{\sigma}}=(I\cos \theta+i\vec{n}\cdot\vec{\sigma}\sin \theta)\]
特殊地
\[e^{i\theta\sigma_z}=(I\cos \theta+i\sigma_z\sin \theta)=\begin{pmatrix}
    e^{i\theta}&0\\
    0&e^{-i\theta}
\end{pmatrix}\]
同时
\[(\vec{a}\cdot\vec{\sigma})(\vec{b}\cdot\vec{\sigma})=\vec{a}\cdot\vec{b}I+i(\vec{a}\times \vec{b})\cdot\vec{\sigma}\]
所以
\[[(\vec{a}\cdot\vec{\sigma}),(\vec{b}\cdot\vec{\sigma})]=2i(\vec{a}\times \vec{b})\cdot\vec{\sigma}\]
\end{document}