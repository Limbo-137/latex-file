\documentclass[UTF8]{ctexart}
%\usepackage{}
\title{内蒙古牧区民宿发展调研报告}
\author{赵翔}
\begin{document}
\maketitle
\begin{abstract}
本文从民宿旅游的概念出发,介绍了目前内蒙古牧区民宿的发展现状及其优缺点。
\end{abstract}

当前我国旅游模式呈现多元化,而民宿旅游作为 一种新兴起的旅游方式在旅游行业中变得越来越重要。
“民宿”源自日本,是仅提供住宿与免费早餐的家庭旅 馆。它不同于传统的旅游住宿,是利用家庭自住资源为 游客提供更加温馨亲切具有当地风土人情的住宿,是新兴的旅游接待方式。

“十二五”期间,我国乡村旅游业共带动了约 10% 的贫困人口脱贫。预计“十三五”期间,乡村旅游接待游客人次年均增长15%,营业收入年均增长18%,每年带动 200万贫困农民脱贫。因此 ,全国很多省 市特别是 中西部地 区都把大力发展乡村旅游与农村 扶 贫 结 合起 来 ,乡村旅游已经成为农村扶贫的重要载体。 “十三五”开始 ,国家从上至下尤其重视以创新手段 引导乡村发展 ,民宿作为乡村记忆 、乡村情感 、乡村愿景的切实载体 ,恰逢其时。

内蒙古牧区的自然和人文文化资源丰富,其中蒙古族饮食文化体验是牧区民宿旅游中的重要服务,厨房是饮食文 化体验 的重要场所。
\section{牧区民宿旅游}
\subsection{牧区民宿旅游的定义}
牧区民宿旅游是指牧民利用自有蒙古包、厨房等工具和没施,通过整体设计和改造,为游 客提 供居住 、 饮 食 、和娱 乐体 验等 服务 的一 种旅游 形式 。牧 区 民宿 旅 游具 有 以下特 征 :
 
(1)地 点位于牧 区,坐拥 草原 自然生 态 ; 

(2)牧民主人的生活是蒙古族典型的生活方式 ;

(3)游客与牧民 主人之 问充满 温情 的交流 、 交心 ,使得游客有归属感与家的体验。
\subsection{牧区民宿旅游与文化体验}
文化旅游,足指旅游者通过对行为习惯、信仰、风俗 、语言 、知识 、艺术 和 自然遗产 等旅 游资 源的认 知 体验 .得 到一种 文 化享受 和 收获 的旅 游 活动[21。牧 区 民宿旅 游 即是一 种文化 旅游 ,牧 区 民宿 游客 的动机 与行为决定了其天然具有文化体验属性 ,是对牧区蒙 古族 典型 生活 方式 的整 体体验。

\end{document}