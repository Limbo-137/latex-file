\documentclass[UTF8]{ctexart}
\usepackage{graphicx}
\usepackage{float}
\usepackage{ragged2e}

\title{内蒙古牧区民宿发展调研报告}
\author{赵翔\footnote{赵翔,内蒙古呼和浩特人,在内蒙古长大,对风土人情有一定的了解。}}
\begin{document}
\maketitle
\begin{abstract}
本文从民宿旅游的概念出发,介绍了目前内蒙古牧区民宿的类型、发展现状及其优缺点,以及一些改进的建议。
\end{abstract}

当前我国旅游模式呈现多元化,而民宿旅游作为 一种新兴起的旅游方式在旅游行业中变得越来越重要。
“民宿”源自日本,是仅提供住宿与免费早餐的家庭旅 馆。它不同于传统的旅游住宿,是利用家庭自住资源为 游客提供更加温馨亲切具有当地风土人情的住宿,是新兴的旅游接待方式。

“十二五”期间,我国乡村旅游业共带动了约 10% 的贫困人口脱贫。预计“十三五”期间,乡村旅游接待游客人次年均增长15%,营业收入年均增长18%,每年带动 200万贫困农民脱贫。因此 ,全国很多省 市特别是 中西部地 区都把大力发展乡村旅游与农村 扶 贫 结 合起 来 ,乡村旅游已经成为农村扶贫的重要载体。 “十三五”开始 ,国家从上至下尤其重视以创新手段 引导乡村发展 ,民宿作为乡村记忆 、乡村情感 、乡村愿景的切实载体 ,恰逢其时。

内蒙古牧区的自然和人文文化资源丰富,其中蒙古族饮食文化体验是牧区民宿旅游中的重要服务,厨房是饮食文 化体验 的重要场所。

\section{牧区民宿旅游的定义}
牧区民宿旅游是指牧民利用自有蒙古包、厨房等工具和没施,通过整体设计和改造,为游 客提 供居住 、 饮 食 、和娱 乐体 验等 服务 的一 种旅游 形式 。牧 区 民宿 旅 游具 有 以下特 征 :
 
(1)地 点位于牧 区,坐拥 草原 自然生 态 ; 

(2)牧民主人的生活是蒙古族典型的生活方式 ;

(3)游客与牧民 主人之 问充满 温情 的交流 、 交心 ,使得游客有归属感与家的体验。
\section{牧区民俗文化旅游服务体验类别}
如图 1,居住文化是民宿旅游中的重要组成部分, 在对居住文化的体验中,游客会体验到牧民的居住设
施———蒙古包、家具等,还会理解到搭建制度、习俗,看 到牧民居住文化以及居住行为。
\begin{figure}[H]
    \centering
    \includegraphics[width=9cm]{20201106_082635.jpeg}
    \caption{内蒙古牧区民宿旅游服务类别}
    \label{fig:my_label}
\end{figure}
\section{牧区民宿旅游与文化体验}
文化旅游,是指旅游者通过对行为习惯、信仰、风俗 、语言 、知识 、艺术 和 自然遗产 等旅 游资 源的认 知 体验 .得 到一种 文 化享受 和 收获 的旅 游 活动[2]。牧 区 民宿旅 游 即是一 种文化 旅游 ,牧 区 民宿 游客 的动机 与行为决定了其天然具有文化体验属性 ,是对牧区蒙 古族 典型 生活 方式 的整 体体验。
\section{牧区文化旅游优缺点}
\textbf{优点}:

1.游客在初来体验牧区蒙古包搭建时,蒙古包的架木结构、捆扎体系、绳索体系与内部布
局机构,对于游客来说是一种充满新鲜感与乐趣的事情,在主人的带领下了解搭建步 骤,以引导者的身份帮助游客体会各部分的功能与作用,获得心理上的愉悦,理解牧区 搭建文化。

2.游客在主人的介绍与指引下,会使用与原有生活用品功能、作用、外形均有所区别的牧 区特色物品,在这个过程中,游客会有一种融入到牧区人民生活中的代入感,提升心理
满足感。

3.游客来到牧区游玩会充满对牧区风景的向往,牧区的草原水草丰美,牛羊成群,一望无 际。置身其中,会被这种美景深深吸引陶醉。

4.来到牧区,游客首先会收到来自民宿主人一家的牧区特色迎客礼仪,之后在民宿体验 了解到问候请安、蒙古包朝、敬酒敬茶敬食、祭祀礼仪等诸多风俗文化体验,诸如此类 均和主人以朋友的关系交流。

5.通过居住牧区传统的蒙古包,游客会获得一个与牧民相通的
居住体验,增强代入感,而蒙古包为了适合牧区生活而打造,游客会体验到这一居住产 品(蒙古包)的各种特点,增强趣味性。

\textbf{缺点}:

1.通过分析城市游客的居住需求,当前产品并未能满 足城市居民的居住需求(舒适性),包括面积空间小、通透性差、室内环境不佳等特点,游客最基本的舒适性 无法得到满足,文化类的体验也不易达到满意效果。

2.当前 ,“牧家乐”层次民宿旅游 中,饮食文化体验 种类较为单一 ,仅仅以品尝蒙古族美食为主

3.在民宿旅游服务体验过程中,牧区传统游牧民族文化可以得以体验,但有些文化因其自身的特性未能很好地得到体现,例如:光影时间概念的提取。在蒙古族传统
文化中,牧民们都是利用阳光运行轨迹而照射蒙古包乌 尼杆的投影来确定时间,此时包与阳光结合形成了一个 日晷,蒙古包这种时间概念功能经历漫长岁月的洗礼而 升华凝结,这是针对传统游牧生活而衍生的文化。而外 来游客或者不够了解其内涵的人群来体验时,因文化与 生活习惯的不同,当事者并不能良好地理解蒙古包传统
光影的真正价值。
\section{建议和总结}

\textbf{建议}:

1.对牧区传统蒙古包的各部分 模块进行整理分析,深度解剖蒙古包的结构,从而抓住其特点,提出针对性改造设计方案,对设计产品进 行模块拼接以及部分改造,从而满足游客在牧区的居住需求。

2.现 阶段“牧家 乐”形式 的饮 食 服务 只停 留在食物 的色香味 ,而其他 的诸 如食物 的 制作技巧,营养搭配等还未被开发出来 ,闳此,应 该 打造多种文化体验服务,使游客能参与到蒙古族的美 食 制作 中去 ,亲 自体 验制 作 奶制 品 ,体验熬 制奶 茶 , 体 验制 作蒙古 族特 色 的食 物等 ,并在其 中 获得文 化认 知 体验 和成 就感 。

3.针对游客忽略光影时间文化的问题,通过改造原有套脑的结构,将之前较难理解的光影改为直观的影像, 用富有情趣的方式体现牧区时间文化,增强游客对此文化的理解,加深认知深度。

\textbf{总结}:

选 择 民宿的背 后 ,是人 们对温情 的渴 望 ,对新奇的东 西 的 向往 。牧 区 民宿 主人具 有 经营者 和服 务提供 者双重身份 ,是淳朴又有人情味的牧民,他们的嘘寒 问暖呵护备至 ,会给游客家一般的感觉。因此 ,民宿 饮食服务中的交互不仅仅是传统的软硬界面交互,更 多的是人与人之间的交互 ,这就要求服务流程 、符号 、 产 品 、环境 的设 计需 要更 多 的考虑沟 通 问题 。

传承是对非物质文化遗产最好的保护手段,而创新 则是在社会变迁、文化生态变化之下拼布艺术由“遗产- 资源”的文化转型的必然之路。传统拼布艺术来自于民 间,现代拼布艺术的良性发展也不能脱离民间和社会的 有用性。“文化资源”的概念是费孝通先生晚年所倡导的
文化遗产保护的理念,对文化遗产进行资源活用,也是
文化遗产从静态转化为活态的过程。也就是说在“文化
资源”的概念中,文化遗产不再只是前人遗留下来的死
去的过去,而是让其成为重振地方文化和地方经济的一 种资源\cite{韩冬楠2018基于民宿旅游视角的牧区蒙古包模块化改造设计}\cite{韩冬楠2018文化体验视角下牧区民宿厨房服务系统设计}。
\bibliographystyle{plain}
\bibliography{references}
\end{document}