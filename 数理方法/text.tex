\documentclass[hyperref,UTF8]{ctexbook}
\usepackage[dvipdfmx]{graphicx}
\usepackage{gbt7714}
\usepackage{float}
\usepackage{ragged2e}
\usepackage{amsthm}
\usepackage{amssymb}
\usepackage{amsmath}
\usepackage{wrapfig}
\usepackage{tikz}
\usetikzlibrary{arrows.meta}
\usepackage{booktabs}
%\usepackage[a4paper,left=3.18cm,right=3.18cm,top=2.54cm,bottom=2.54cm]{geometry}
\usepackage{tabularx}
\usepackage{array}
\usepackage{caption}
\usepackage{hyperref}
\hypersetup{hypertex=true,
            colorlinks=true,
            linkcolor=red,
            anchorcolor=blue,
            citecolor=red}
\newcommand{\upcite}[1]{\textsuperscript{\textsuperscript{\cite{#1}}}}
\newtheorem{eg}{例}[chapter]
\newtheorem{theorem}{定理}[chapter]
\setCJKfamilyfont{song}{SimSun}
\title{现代数学物理教程\\群,希尔伯特空间和微分几何}
\author{Peter Szekeres\\limbo137译}
\begin{document}
\maketitle
\tableofcontents
\chapter{集合和结构}
(填充)
\section{集合和逻辑}
(填充)
\section{子集,集合的交和并}
(填充)
\subsection{交和并}
(填充)
\section{笛卡尔积和关系}
(填充)
\subsection{有序对和笛卡尔积}
(填充)
\subsection{关系}
(填充)
\subsection{等价关系}
(填充)
\subsection{序关系和偏序集合}
(填充)
\section{映射}
(填充)
\subsection{单射,满射和双射}
(填充)
\section{无限集}
(填充)
\subsection{可数集}(填充)
\subsection{不可数集}(填充)
\subsection{连续统和选择公理}(填充)
\section{结构}(填充)
\subsection{代数结构}(填充)
\subsection{几何结构}(填充)
\subsection{动力系统}(填充)
\section{范畴论}(填充)
\chapter{群}
\section{群论的元素}
集合 $G$上任意一对 $g,h\in G$,都存在 $gh\in G$称为其\textbf{乘积}(product),满足
\begin{enumerate}
    \item[(Gp1)]\label{gp:1}
    \textbf{结合律}(associative law): 对全体 $g,h,k\in G$有$g(hk)=(gh)k$  
    \item[(Gp2)] \label{gp:2}
    存在\textbf{恒等元}(identity element): $e\in G$,使得对全体\(g\in G\)
    \[eg=ge=g\]
    \item[(Gp3)] \label{gp:3}
    每个群元 $g\in G$有\textbf{逆}(inverse) $g^{-1}\in G$ 使得\[g^{-1}g=gg^{-1}=e\] 
\end{enumerate}
这样的集合称为\textbf{群}(group)

对任意两个元素都存在乘积,且乘积在群内,这个性质称为\textbf{封闭性}(closure property)

类似的,可证明每个$g\in G$ 都有唯一逆 $g^{-1}$且 $(gh)^{-1}=h^{-1}g^{-1}$ 
\\
如果群元可交换,即\[gh=hg\]总成立,则称群是\textbf{阿贝尔的}(abelian),如果群乘法写成加法的形式,那我们默认其是阿贝尔的, $g$ 的逆是$-g$ 

集合 $H\subseteq G$满足:
\begin{enumerate}
    \item[(a)] $h,k\in H\Rightarrow hk\in H$ 
    \item[(b)] $e\in H$ 
    \item[(c)] $h\in H,h^{-1}\in H$ 
\end{enumerate} 
$H$称为 $G$的子群,无需结合律(Gp1),因为结合律自动继承下来了  
\begin{eg}
    全体实数和加法形成一个群,称为实数加法群(additive group of reals). $e=0$,$x$的逆是 $-x$ ,整数加法群和有理数加法群显然都是 $\mathbb{R}$的子群 
\end{eg}
\begin{eg}
    非零实数$\dot{\mathbb{R}}=\mathbb{R}-\{0\}$形成一个群,称作实数乘法群(multiplication group of reals),群乘法是普通乘法,恒等元$e=1$
    类似的,$\mathbb{\dot{Q}}$也是$\mathbb{\dot{R}}$的子群
    $\mathbb{C}$上同样有加法群,$\mathbb{\dot{C}}$上同样有乘法群
\end{eg}
\section{变换和排列群}(填充)
\section{矩阵群}
\subsection{线性变换}
$\mathbb{R}^{n}$是 $n\times 1$实列向量空间
\[\vec{x}=\begin{pmatrix}
    x_1\\x_2\\x_3\\\vdots\\x_n
\end{pmatrix}\]  
\subsection{矩阵群}
全体$n\times n$非奇异方阵组成一个群,记作 $GL(n,\mathbb{R})$ 原因在于行列式乘积法
\[\det(AB)=det(A)det(B)\] 
利用这一点可证明非奇异方阵可以组成一个群,满足封闭性,结合律以及可逆

在复数情况也可以做类似的讨论,记作$GL(n,\mathbb{C})$,与上面的实数情况并称为\textbf{n阶一般线性群}(general linear group of order n),其子群即群乘法为矩阵乘法的,统称为\textbf{矩阵群}(matrix groups)
\begin{eg}
    $n$阶的特殊线性群(special liner group)或称幺模群(unimodular group)记作$SL(n,\mathbb{R})$
\end{eg}
\section{同态和同构}
\subsection{同态}
两个群$G$和$G'$,映射$\varphi : G \mapsto G'$保持群乘法,$$
\varphi(a b)=\varphi(a) \varphi(b)
$$
则称这个映射是同态 (homomorphism)
\begin{theorem}
    在同态映射下$G$的单位元映射到$G'$的单位元,任意元素之逆映为像元素之逆
\end{theorem}
\begin{eg}
    对任意实数$x\in \mathbb{R}$整数部分(integral part)记成$[x]$,小数部分(fractional part)写成$(x)=x-[x]$,显然$0 \leqslant(x)<1$\\
    在$[0,1]$区间上定义模\(1\)加法$$
    a+b \bmod 1=(a+b)
    $$
    这个群是阿贝尔的,称为实数模\(1\)群,\(a\)的逆元是\(1-a\),\(0\)的逆元是\(0\)
\end{eg}

\subsection{同构}(填充)
\subsection{自同构和共轭类}(填充)
\section{正规子群和商群}(填充)
\subsection{陪集}(填充)
\subsection{正规子群}(填充)
\subsection{商群}(填充)
\subsection{同态的核}(填充)
\section{群作用}(填充)
\section{对称群}(填充)
\chapter{向量空间}(填充)
\section{环和域}(填充)
\section{向量空间}(填充)
\section{向量空间的同态}(填充)
\section{向量空间的子空间和商空间}(填充)
\subsection{互补子空间和商空间}(填充)
\subsection{线性映射的像和核}(填充)
\section{向量空间的基}(填充)
\subsection{集合张成的子空间}(填充)
\subsection{向量空间的基}(填充)
\subsection{线性算符的矩阵}(填充)
\subsection{基扩张定理}(填充)
\section{求和惯例和基变换}(填充)
\subsection{求和惯例}(填充)
\subsection{基变换}(填充)
\section{对偶空间}(填充)
\subsection{线性泛函}(填充)
\subsection{线性空间的对偶空间}(填充)
\subsection{对偶的对偶}(填充)
\subsection{余向量分量的变换法则}(填充)
\chapter{线性算符和矩阵}(填充)
\section{本征空间和特征方程}(填充)
\subsection{不变子空间}(填充)
\subsection{本征向量和本征值}(填充)
\subsection{特征方程}(填充)
\subsection{最小零化多项式}(填充)
\section{约当标准形}(填充)
\subsection{分块对角形}(填充)
\subsection{幂零算子}(填充)
\subsection{约当标准形}(填充)
\section{线性常微分方程}(填充)
\subsection{二维自洽体系}(填充)
\section{群表示论简介}(填充)
\subsection{不可约表示}(填充)
\subsection{舒尔引理}(填充)
\chapter{内积空间}(填充)
\section{实内积空间}(填充)
\subsection{实内积的分量}(填充)
\subsection{正交标准基}(填充)
\section{复内积空间}(填充)
\subsection{向量的范数}(填充)
\subsection{正交标准基}(填充)
\subsection{幺正变换}(填充)
\section{有限群的表示}(填充)
\subsection{正交关系}(填充)
\chapter{代数}(填充)
\section{代数和理想}(填充)
\subsection{理想和商代数}(填充)
\section{复数和复结构}(填充)
\subsection{实向量空间的复化}(填充)
\subsection{向量空间的复结构}(填充)
\section{四元数和Clifford代数}(填充)
\subsection{四元数}(填充)
\subsection{Clifford代数}(填充)
\section{Grassmann代数}(填充)
\subsection{多重向量}(填充)
\subsection{外积}(填充)
\subsection{外积的性质}(填充)
\section{李群和李代数}(填充)
\subsection{矩阵李群}(填充)
\subsection{单参数子群}(填充)
\subsection{复李代数}(填充)
\chapter{张量}(填充)
\section{自由向量空间和张量空间}(填充)
\subsection{自由向量空间}(填充)
\subsection{张量基}(填充)
\subsection{张量积的对偶表示}(填充)
\subsection{自由结合代数}(填充)
\subsection{Grasmann代数作为自由代数的商代数}(填充)
\section{多线性映射和张量}(填充)
\subsection{多重线性映射和$(r,s)$型张量}(填充)
\subsection{二阶协变张量}(填充)
\subsection{二阶逆变张量}(填充)
\subsection{混合张量}(填充)
\section{张量的基表示}(填充)
\subsection{张量积}(填充)
\subsection{基变换}(填充)
\section{张量的作用}(填充)
\subsection{缩并}(填充)
\subsection{升高降低指标}(填充)
\subsection{对称性}(填充)
\chapter{外代数}(填充)
\section{r-向量和r-形式}(填充)
\subsection{算符A的反对称化}(填充)
\section{r-向量的基表示}(填充)
\section{外积}(填充)
\subsection{简单p-向量及其子空间}(填充)
\section{内积}(填充)
\section{定向的向量空间}(填充)
\subsection{n-向量和n-形式}(填充)
\subsection{n-向量和n-形式的变换法则}(填充)
\subsection{有向的向量空间}(填充)
\subsection{$\epsilon$-符号}(填充)
\section{Hodge对偶}(填充)
\subsection{p-向量的内积}(填充)
\subsection{Hodge星算子}(填充)
\chapter{狭义相对论}(填充)
\section{闵可夫斯基时空}(填充)
\subsection{庞加莱和洛伦兹变换}(填充)
\subsection{仿射几何}(填充)
\subsection{闵可夫斯基空间和4-张量}(填充)
\section{狭义相对论运动学}(填充)
\subsection{狭义洛伦兹变换}(填充)
\subsection{时间,张度,速度的相对性}(填充)
\section{粒子动力学}(填充)
\subsection{世界线和固有时}(填充)
\subsection{相对论性粒子动力学}(填充)
\section{电动力学}(填充)
\subsection{4-张量场}(填充)
\subsection{电磁学}(填充)
\subsection{势和规范变换}(填充)
\section{守恒定律和能动张量}(填充)
\subsection{电荷守恒}(填充)
\subsection{能动张量}(填充)
\chapter{拓扑}(填充)
\section{欧氏拓扑}(填充)
\section{广义拓扑空间}(填充)
\section{矩阵空间}(填充)
\section{诱导拓扑}(填充)
\subsection{诱导拓扑和拓扑乘积}(填充)
\subsection{同化拓扑}(填充)
\section{豪斯多夫空间}(填充)
\section{紧空间}(填充)
\section{连通空间}(填充)
\section{拓扑群}(填充)
\subsection{单位元的连通分量}(填充)
\section{拓扑向量空间}(填充)
\subsection{巴拿赫空间}(填充)
\chapter{测度论和积分}(填充)
\section{可测空间和函数}(填充)
\subsection{可测空间}(填充)
\subsection{可测函数}(填充)
\section{测度空间}(填充)
\subsection{勒贝格测度}(填充)
\section{勒贝格积分}(填充)
\subsection{勒贝格控制收敛定理}(填充)
\chapter{分布}(填充)
\section{测试函数和分布}(填充)
\subsection{测试函数空间}(填充)
\subsection{分布}(填充)
\subsection{正规分布}(填充)
\section{分布上的算符}(填充)
\subsection{分布的微分}(填充)
\subsection{$\delta$-函数的变量代换}(填充)
\section{傅立叶变换}(填充)
\section{格林函数}(填充)
\subsection{柏松方程}(填充)
\subsection{波动方程的格林函数}(填充)
\chapter{希尔伯特空间}(填充)
\section{定义和例子}(填充)
\section{扩张定理}(填充)
\subsection{子空间}(填充)
\subsection{正交标准基}(填充)
\section{线性泛函}(填充)
\subsection{正交子空间}(填充)
\subsection{Riesz表示定理}(填充)
\section{有界线性算符}(填充)
\subsection{伴随算符}(填充)
\subsection{哈密顿算符}(填充)
\subsection{幺正算符}(填充)
\section{谱理论}(填充)
\subsection{本征向量}(填充)
\subsection{有界算符的谱}(填充)
\subsection{哈密顿算符的谱}(填充)
\section{无界算符}(填充)
\subsection{自伴和对称算符}(填充)
\subsection{无界算符的谱}(填充)
\chapter{量子力学}(填充)
\section{基本概念}(填充)
\subsection{光子偏振实验}(填充)
\subsection{态的希尔伯特空间}(填充)
\subsection{可观测量}(填充)
\subsection{量子力学中的无界算符}(填充)
\section{量子动力学}(填充)
\subsection{海森堡图景}(填充)
\subsection{经典力学和波动力学的对应关系}(填充)
\subsection{谐振子}(填充)
\subsection{角动量}(填充)
\section{对称变换}(填充)
\subsection{无穷小生成元}(填充)
\subsection{粒子数守恒}(填充)
\subsection{离散对称性}(填充)
\subsection{全同粒子}(填充)
\section{量子统计力学}(填充)
\subsection{密度算符}(填充)
\subsection{系综}(填充)
\subsection{全同粒子体系}(填充)
\chapter{微分几何}(填充)
\section{可微流形}(填充)
\section{可微映射和曲线}(填充)
\section{切空间,余切空间和张量空间}(填充)
\subsection{切矢量}(填充)
\subsection{余切空间和张量空间}(填充)
\subsection{向量和张量场}(填充)
\subsection{坐标变换}(填充)
\subsection{张量丛}(填充)
\section{切映射和子流形}(填充)
\subsection{切映射和映射的拉回}(填充)
\subsection{子流形}(填充)
\section{对易子,流和李导数}(填充)
\subsection{对易子}(填充)
\subsection{积分曲线和流}(填充)
\section{分布和Frobenius定理}(填充)
\chapter{可微形式}(填充)
\section{微分形式和外导数}(填充)
\section{外导数的性质}(填充)
\section{Frobenius定理:对偶形式}(填充)
\section{热力学}(填充)
\subsection{热力学第二定律}(填充)
\subsection{绝对熵和温度}(填充)
\section{经典力学}(填充)
\subsection{变分学}(填充)
\subsection{拉格朗日力学}(填充)
\subsection{哈密顿力学}(填充)
\subsection{拉格朗日力学和哈密顿力学之间的联系}(填充)
\chapter{流形上的积分}(填充)
\section{单位分解}(填充)
\section{n-形式的积分}(填充)
\section{stokes定理}(填充)
\subsection{常规定义域}(填充)
\section{同调和上同调}(填充)
\subsection{有序单纯形和欧氏空间中的链}(填充)
\subsection{流形上的单纯同调}(填充)
\subsection{De Rham上同调群和对偶性}(填充)
\section{庞加莱引理}(填充)
\subsection{电动力学}(填充)
\chapter{联络和曲率}(填充)
\section{线性联络和测地线}(填充)
\subsection{坐标变换}(填充)
\section{张量场的协变导数}(填充)
\section{曲率和挠率}(填充)
\subsection{挠率张量}(填充)
\subsection{曲率张量}(填充)
\section{伪黎曼流形}(填充)
\subsection{黎曼联络}(填充)
\subsection{测地坐标}(填充)
\section{测地偏离方程}(填充)
\section{黎曼张量及其对称性}(填充)
\subsection{比安基恒等式}(填充)
\section{嘉当标准形}(填充)
\subsection{嘉当标准形中的伪黎曼空间}(填充)
\subsection{局域平直空间}(填充)
\section{广义相对论}(填充)
\subsection{等效原理}(填充)
\subsection{广义相对论的基本假设}(填充)
\subsection{曲率张量的测量}(填充)
\subsection{线性近似}(填充)
\subsection{史瓦西解}(填充)
\section{宇宙学}(填充)
\section{时空的变分原理}(填充)
\subsection{希尔伯特作用量}(填充)
\subsection{场的能动张量}(填充)
\chapter{李群和李代数}(填充)
\section{李群}(填充)
\subsection{左不变向量场}(填充)
\subsection{李群的李代数}(填充)
\subsection{Maurer-Cartan关系}(填充)
\section{指数映射}(填充)
\subsection{指数映射}(填充)
\section{李子群}(填充)
\subsection{矩阵李群}(填充)
\section{李群的变换}(填充)
\subsection{正规子群}(填充)
\section{保度规群}(填充)
\subsection{球对称}(填充)


\end{document}